\section*{پیش‌گفتار}
\addcontentsline{toc}{section}{پیش‌گفتار}
در علوم کامپیوتر درس‌هایی وجود دارند که از آنها به عنوان درس‌هایی بنیادین یاد می‌شود. برخی از این درس‌ها عبارت‌اند از: داده‌ساختارها، تجزیه و تحلیل الگوریتم‌ها، طراحی کامپایلر و نظریه‌ی زبان‌ها و ماشین‌ها.

از میان این درس‌های بنیادین یکی از مهمترین آنها، درس داده‌ساختارها است که به عنوان پیشنیازی برای درس‌هایی همچون تجزیه و تحلیل الگوریتم‌ها و سیستم‌های عامل نیز مطرح است. اگر داده‌ساختار را به این صورت تعریف کنیم که {\prq}یک داده‌ساختار روشی برای ذخیره و سازماندهی داده‌ها است به طوریکه بازیابی و/یا تغییر داده‌ها به سادگی و با کارایی بالا انجام شود{\plq} آنگاه می‌توان به تعریفی از درس داده‌ساختارها نیز رسید. در درس داده‌ساختارها به بررسی دقیق و موشکافانه‌ی انواع داده‌ساختارها و چگونگی پیاده‌سازی آنها در یک زبان برنامه‌نویسی پرداخته می‌شود.

به دلیل اهمیت درس داده‌ساختارها کتاب‌های مختلفی در مورد آن نوشته شده است که بسیاری از آنها دارای قالب کم و بیش یکسانی هستند. قالب کلی این کتاب‌ها به این شکل است که در هر فصل از کتاب ابتدا به معرفی و بررسی یک داده‌ساختار خاص پرداخته شده و در انتهای فصل تمریناتی مرتبط با آن داده‌ساختار ارائه می‌شود. در کتاب حاضر سعی شده است از قالبی متفاوت استفاده شود.

در این کتاب فرض بر این است که خواننده با مباحث مختلف داده‌ساختارها آشنایی نسبی دارد و در نتیجه هر فصل از این کتاب دارای بخش نخست کتابهای معمول، یعنی معرفی و بررسی یک داده‌ساختار، نیست. تمرکز این کتاب بر روی مطرح کردن تعدادی سوال در مورد هر یک از انواع داده‌ساختارها و دادن پاسخ گام به گام و تشریحی به هر یک از سوالات است. به بیانی دیگر می‌توان قالب این کتاب را به صورت پرسش و پاسخ در نظر گرفت که به خواننده کمک می‌کند تا فهم عمیقتری از داده‌ساختارهای مختلف به دست آورد.

در نگارش حاضر، فصل‌های اول، دوم، سوم و پنجم در کتاب گنجانده شده‌اند و فصل چهارم پس از آماده‌سازی و کسب اطمینان از کیفیت علمی و ظاهری آن در نگارش‌ بعدی به کتاب افزوده خواهد شد.

متن کتاب با استفاده از سیستم حروفچینی {\lr{\LaTeX}}، بسته‌ی {\lr{\XePersian}} و ویرایشگر {\lr{bidiTexmaker}} آماده شده است. برای حروفچینی متن پارسی از قلم {\lr{XB Niloofar}} و برای حروفچینی کلمات انگلیسی و شبه‌کدها از قلم {\lr{Computer Modern}} استفاده شده است. طراحی جلد و شکل‌های کتاب با استفاده از نرم‌افزار {\lr{Corel DRAW}} انجام شده است. برای دسترسی به متن خام کتاب می‌توانید به نشانی اینترنتی {\lr{\url{https://github.com/MasoodFallahpoor/DS-Book}}} مراجعه کنید.

در آماده‌سازی این کتاب تلاش شده است تا چه از نظر علمی و چه از نظر ظاهری کتابی شایسته و خالی از خطا به خوانندگان تقدیم شود. اما از آنجایی که هیچ کتابی نمی‌تواند به طور کامل از خطا در امان باشد از این رو از شما خواننده‌ی گرامی خواهشمندم در صورت مشاهده هرگونه اشتباه املایی، نگارشی و یا علمی به نشانی
{\lr{\href{mailto:masood.fallahpoor@gmail.com}{\texttt{masood.fallahpoor@gmail.com}}}} اطلاع دهید تا در نگارش‌های بعدی برطرف شود.

\begin{flushleft}
\author
\end{flushleft}
\vskip -1ex%
{\newlength{\authorwidth}%
\setlength{\authorwidth}{\widthof{\author}}%
\hfill\hbox to \authorwidth {\date\hfill}%
}
\newpage