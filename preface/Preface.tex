\section*{پیش‌گفتار}
\addcontentsline{toc}{section}{پیش‌گفتار}
در علم کامپیوتر درس‌هایی وجود دارند که از آنها به عنوان درس‌های بنیادین این علم یاد می‌شود. برخی از این درس‌ها عبارت‌اند از: داده‌ساختارها، تجزیه و تحلیل الگوریتم‌ها، طراحی کامپایلر، نظریه‌ی زبان‌ها و ماشین‌ها و سیستم‌های عامل. فهم عمیق و تسلط بر هر یک از این درس‌ها کمک شایانی به یادگیری راحتتر سایر درس‌ها خواهد کرد.

از میان درس‌های بنیادین علم کامپیوتر یکی از مهمترین آنها، درس داده‌ساختارها است که به عنوان پیشنیازی برای درس‌هایی همچون تجزیه و تحلیل الگوریتم‌ها و سیستم‌های عامل نیز مطرح است. اگر داده‌ساختار را به این صورت تعریف کنیم که {\prq}داده‌ساختار روشی برای ذخیره و سازماندهی داده‌ها است به طوریکه بازیابی و/یا تغییر داده‌ها به سادگی و با کارایی بالا انجام شود{\plq} آنگاه می‌توان به تعریفی از درس داده‌ساختارها نیز رسید. در درس داده‌ساختارها به بررسی دقیق و موشکافانه‌ی انواع داده‌ساختارها و چگونگی پیاده‌سازی آنها در یک زبان برنامه‌نویسی پرداخته می‌شود.

به دلیل اهمیت درس داده‌ساختارها کتاب‌های مختلفی در مورد آن نوشته شده است که اکثر آنها دارای قالب کم و بیش یکسانی هستند. قالب کلی این کتاب‌ها به این شکل است که در هر فصل از کتاب ابتدا به معرفی و بررسی یک داده‌ساختار خاص پرداخته می‌شود و در انتهای فصل تمریناتی مرتبط با آن داده‌ساختار ارائه می‌شود. در کتاب حاضر سعی شده است از قالبی متفاوت و جدید استفاده شود.

در این کتاب فرض بر این است که خواننده با مباحث مختلف داده‌ساختارها آشنایی نسبی دارد و در نتیجه هر فصل از این کتاب دارای بخش نخست کتابهای معمول، یعنی معرفی و بررسی یک داده‌ساختار، نیست. تمرکز این کتاب بر روی مطرح کردن تعدادی سوال در مورد هر یک از انواع داده‌ساختارها و دادن پاسخ تشریحی به هر یک از این سوالات است. به بیانی دیگر می‌توان قالب این کتاب را به صورت پرسش و پاسخ در نظر گرفت که به خواننده کمک می‌کند تا فهم عمیقتری از داده‌ساختارهای مختلف به دست آورد.

متن کتاب با استفاده از سیستم حروفچینی لاتک ، بسته‌ی زی‌پرشین و ویرایشگر {\lr{bidiTexmaker}} آماده شده است. برای متن پارسی از قلم {\lr{XB Niloofar}} و برای کلمات انگلیسی و شبه‌کدها از قلم {\lr{Computer Modern}} استفاده شده است. برای طراحی جلد کتاب از نرم‌افزار {\lr{Corel DRAW}} و برای رسم شکل‌های کتاب از بسته‌ی {\lr{PSTricks}} و نرم‌افزار {\lr{LaTeXDraw}} استفاده شده است. برای دسترسی به متن خام کتاب می‌توانید به نشانی {\lr{\url{https://github.com/MasoodFallahpoor/DS-Book}}} مراجعه کنید.

در آماده‌سازی این کتاب تلاش شده است تا چه از نظر علمی و چه از نظر ظاهری کتابی شایسته و خالی از خطا به خوانندگان تقدیم شود. اما از آنجایی که هیچ کتابی نمی‌تواند به طور کامل از خطا در امان باشد از این رو از شما خواننده‌ی گرامی خواهشمندم در صورت مشاهده هرگونه خطای املایی، نگارشی و یا علمی به نشانی رایانامه
{\lr{\href{mailto:masood.fallahpoor@gmail.com}{\texttt{masood.fallahpoor@gmail.com}}}} اطلاع دهید تا خطای موجود در نگارش‌های بعدی کتاب رفع شود.

\begin{flushleft}
\small
مسعود فلاح‌پور\\
مهر ۱۳۹۳
\end{flushleft}