% AMS packages for typesetting mathematics more effectively.
\usepackage{amsthm}
\usepackage{amsmath}
\usepackage{amsfonts}
\usepackage{amssymb}

\allowdisplaybreaks[4]

\usepackage{pifont}

% This package provides extensible arrows
\usepackage{extarrows}

% Set the math indent to zero centimeter so that there is no 
% leading space between left margin and display mode equations.
\setlength{\mathindent}{0cm}

% Set the margins
\usepackage[headheight=17.0pt,top=3cm, bottom=3cm, left=3cm, right=3cm]{geometry}

% This packege is used to customize the numbering scheme of "enumerate" environment.
\usepackage{enumerate}

% This package provides facilities to have more control over float environments.
\usepackage{float}

% By using this package the paragraph indentation is set to zero and
% some vertical space is added between paragraphs.
\usepackage{parskip}

% This package provides facilities for drawing various shapes.
\usepackage{pstricks}

% This package is used to customize the caption of float environments.
\usepackage[labelformat=parens,labelfont={bf},margin=1cm,font=footnotesize]{caption}

% This package is used to reset footnote counter for each new page.
\usepackage{zref-perpage} 
\zmakeperpage{footnote}

% These packages provide facilities for typesetting algorithms.
\usepackage{algorithm}
\usepackage{algpseudocode}

% This package provides the ability to change the dependency of counters.
\usepackage{chngcntr}

% Counter "algorithm" is reset to zero each time counter "chapter" is incremented.
\counterwithin{algorithm}{chapter}

% This package is used to increase the spacing between lines of a paragraph
\usepackage[onehalfspacing]{setspace}

\usepackage{fancyhdr}

\usepackage[linewidth=0.7pt]{mdframed}

\usepackage{pdfpages}

\usepackage{tocbibind}

\usepackage[font=footnotesize]{subfig}

\usepackage{hyperref}
\hypersetup{
	colorlinks=true,
	allcolors=blue,
	pdftitle={Questions and Answers of Data Structures},
	pdfauthor={Masood Fallahpoor}
}

% This package provides the ability to perform infix notation arithmatic in LaTeX
\usepackage{calc}

% Provides publication quality tables
\usepackage{booktabs}

% Package for typesetting in Persian.
\usepackage{xepersian}

% Set the default font of Persian text.
\settextfont[Scale=1]{XB Niloofar}

% Reduce the default size of Persian digits in math mode.
\setdigitfont[Scale=1]{XB Niloofar}

\defpersianfont\nastaliq[Scale=2]{IranNastaliq}

\makeatletter

% Change the numbering of algorithms.
% For example the first algorithm in second chapater is numbered as 1.2
\renewcommand{\thealgorithm}{\thechapter\@SepMark\arabic{algorithm}}

%\def\abj@num@i#1{%
%	\ifcase#1
%		\or الف%
%		\or ب%
%		\or ج%
%		\or د%
%		\or ه%
%		\or و%
%		\or ز%
%		\or ح%
%		\or ط%
%	\fi%
%	\ifnum#1=\z@\abjad@zero\fi%
%}
\makeatother

%*********************************************
%*                   Definition of macros     		   		    *
%*********************************************
% A new counter, with initial value of 0, is defined to automate the numbering of questions.
% This counter is reset to 0 everytime counter "chapter" is incremented.
\newcounter{qcnt}[chapter]

% A new counter, with initial value of 0, is defined to automate the numbering of answers.
% This counter is reset to 0 everytime counter "chapter" is incremented.
\newcounter{acnt}[chapter]

% This macro typesets the header of a question.
\فرمان‌نو{\سوال}{\گام‌شمارنده{qcnt}\بدون‌تورفتگی{\scalebox{0.75}{$\blacktriangleleft$}}\فضا\متن‌سیاه{سوال {\theqcnt}.\فضا}}
% This macro typesets the header of an answer.
\فرمان‌نو{\پاسخ}{\گام‌شمارنده{acnt}\بدون‌تورفتگی{\scalebox{0.75}{$\lhd$}}\فضا\متن‌سیاه{پاسخ سوال {\theacnt}.\فضا}}

% macros for typesetting attribute(s) of objects.
\newcommand{\attribxi}[2]{\ensuremath{#1.\hspace*{1pt}\id{#2}}}
\newcommand{\attribxx}[2]{\ensuremath{#1.\hspace*{1pt}#2}}
\newcommand{\attribix}[2]{\ensuremath{\id{#1}\hspace*{1pt}.#2}}
\newcommand{\attribii}[2]{\ensuremath{\id{#1}\hspace*{1pt}.\hspace*{1pt}\id{#2}}}
\newcommand{\attrib}[2]{\attribxi{#1}{#2}}
\newcommand{\attribe}[3]{\attribxi{(#1,#2)}{#3}}
\newcommand{\attribex}[3]{\attribxx{(#1,#2)}{#3}}
\newcommand{\attribb}[3]{\attribxi{\attribxi{#1}{#2}}{#3}}
\newcommand{\attribbb}[4]{\attribxi{\attribb{#1}{#2}{#3}}{#4}}
\newcommand{\attribbbb}[5]{\attribxi{\attribbb{#1}{#2}{#3}{#4}}{#5}}
\newcommand{\attribbxxi}[3]{\attribxi{\attribxx{#1}{#2}}{#3}}

% change comment symbol to "//"
\algrenewcommand{\algorithmiccomment}[1]{\texttt{\textbf{/\hspace*{-0.2em}/}} #1}

\algrenewcommand\algorithmicdo{}
\algrenewcommand\algorithmicthen{}

% defenition of some keywords
\newcommand{\To}{\ensuremath{\mathbf{\ to\ }}}
\newcommand{\Downto}{\ensuremath{\mathbf{\ downto\ }}}
\renewcommand{\And}{\ensuremath{\mathbf{\ and\ }}}
\newcommand{\Or}{\ensuremath{\mathbf{\ or\ }}}

% To typeset equality operator in conditionals
\DeclareMathOperator{\isequal}{\scalebox{0.8}[1]{=}\hspace*{1pt}\scalebox{0.8}[1]{=}}
%\newcommand{\isequal}{\mathrel{\scalebox{0.8}[1]{=}\hspace*{1pt}\scalebox{0.8}[1]{=}}}

% To typeset subarray ranges.
\newcommand{\twodots}{\mathinner{\ldotp\ldotp}}

% To typeset identifiers that are more than one letter
\newcommand{\id}[1]{\ensuremath{\mathit{#1}}}

\newcommand{\oftype}{\ensuremath{\,:\,}}

\newcommand{\version}[2]{%
نگارش {#2\scalebox{0.8}{/}{#1}}
}

\renewcommand{\date}{%
آبان ۱۳۹۳
}

\renewcommand{\author}{%
مسعود فلاح‌پور%
}

% Override "gets" macro; changing it from left-pointing arrow to =
\renewcommand{\gets}{\mathrel{\hspace{1pt}=\hspace{1pt}}}

\makeatletter

% To typeset constants like TRUE,FALSE and NULL
\newcommand{\const}[1]{%
	\if@RTL%
		\lr{\ensuremath{\textsc{#1}}}%
	\else%
		\ensuremath{\textsc{#1}}%
	\fi%
}

\newcommand{\bcall}[2]{%
	\if@RTL%
		\lr{\Call{#1}{#2}}%
	\else%
		\Call{#1}{#2}%
	\fi%
}

\makeatother

% Set the bullet symbols of itemize environment
\renewcommand{\labelitemi}{\scalebox{0.6}{\ding{108}}}
\renewcommand{\labelitemii}{\scalebox{0.6}{\ding{110}}}