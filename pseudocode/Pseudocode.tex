\section*{قواعد شبه‌کد}
\addcontentsline{toc}{section}{قواعد شبه‌کد}

برای بیان الگوریتم‌های بیان شده در کتاب، به جای استفاده از یک زبان برنامه‌نویسی خاص، از شبه‌کد استفاده شده است. با استفاده از شبه‌کد می‌توان الگوریتم‌ها را به شکلی ساده بیان کرد و از بیان جزئیات غیر ضروری خودداری کرد. در ادامه توضیحاتی در مورد کلیات شبه‌کد استفاده شده در کتاب بیان خواهد شد.

\زیرقسمت*{توضیحات}
اگر در قسمتی از شبه کد نیاز به توضیح وجود داشته باشد، مانند زبان {\lr{C++}} از دو علامت اسلش پشت سرهم برای شروع توضیح استفاده می‌شود. در ادامه نمونه‌ای از یک توضیح آورده شده است.
\begin{latin}
\Comment{This is a comment}
\end{latin}

\زیرقسمت*{زیربرنامه‌ها}
تمامی الگوریتم‌های کتاب به صورت زیربرنامه تعریف می‌شوند. یک زیربرنامه دارای دو نوع است: تابع و رویه. اگر زیربرنامه بخواهد مقداری را به عنوان خروجی بازگرداند آنگاه از تابع استفاده می‌کنیم و اگر مقداری را برنگرداند از رویه استفاده خواهیم کرد. 

تعریف یک تابع با کلمه کلیدی {\lr{function}} آغاز می‌شود. سپس نام تابع بیان می‌شود و در صورتی که تابع دارای ورودی باشد، ورودی‌های تابع در داخل پرانتز آورده می‌شوند. در ادامه‌ی تعریف تابع، بدنه تابع شروع می‌شود و در انتها مقداری به عنوان خروجی تابع توسط دستور {\lr{return}} برگشت داده می‌شود. عبارت {\lr{end function}} نیز خاتمه تعریف تابع را نشان می‌دهد. شکل کلی تعریف یک تابع در ادامه نشان داده شده است.
\begin{latin}
\begin{algorithmic}[1]
\Function{‌FunctionName}{param1, param2, $\ldots$ , paramN}
	\State	\Comment{body of function}
	\State	\Return result
\EndFunction
\end{algorithmic}
\end{latin}
شکل کلی تعریف یک رویه هم مانند یک تابع است با این تفاوت‌ها که تعریف یک رویه با کلمه کلیدی {\lr{procedure}} آغاز می‌شود، مقداری توسط رویه بازگردانده نمی‌شود و همچنین خاتمه رویه توسط عبارت {\lr{end procedure}} مشخص می‌شود.

\زیرقسمت*{متغیرها}
برای تعریف متغیرها از قالب {\lr{$\id{VarName} \oftype \id{VarType}$}} استفاده می‌شود. برای مثال اگر بخواهیم متغیر {$i$} را از نوع عدد صحیح تعریف کنیم خواهیم نوشت {\lr{$i \oftype \id{integer}$}}.

متغیرهای مورد استفاده در شبه‌کد‌ها در اکثر مواقع به صورت صریح تعریف نمی‌شوند و فرض بر این است که با اولین استفاده از یک متغیر، آن متغیر به صورت ضمنی تعریف نیز می‌شود. در حالت تعریف ضمنی متغیرها، با توجه به شبه‌کدی که متغیر در آن مورد استفاده قرار گرفته است، به راحتی می‌توان به نوع آن نیز پی برد.

\زیرقسمت*{آرایه‌ها}
اندیس تمامی آرایه‌ها از عدد یک آغاز می‌شود مگر اینکه در یک شبه‌کد صراحتاً چیز دیگری بیان شود. برای دسترسی به خانه‌‌ی {$i$} ام آرایه یک بعدی {$A$} از قالب {$A[i]$} و برای دسترسی به عنصر سطر {$i$} ام و ستون {$j$} ام آرایه دو بعدی {$B$} از قالب {$B[i,j]$} استفاده می‌شود. 

اگر متغیر {$A$} نشان دهنده یک آرایه یک بعدی باشد آنگاه طول این آرایه در خصیصه {\lr{\textit{length}}} آن قرار دارد و برای دسترسی به آن از قالب {$\attrib{A}{length}$} استفاده می‌شود. اگر {$A$} یک آرایه دو بعدی باشد تعدادی سطرهای آن در خصیصه {\lr{\textit{row}}} و تعداد ستون‌های آن در خصیصه {\lr{\textit{column}}} قرار دارد و برای دسترسی به آنها به ترتیب از قالب {$\attrib{A}{rows}$} و {$\attrib{A}{columns}$} استفاده می‌شود. 

جهت اشاره به بازه‌ای از یک آرایه از قالب {$A[i\twodots j]$} استفاده می‌شود که در آن {$i$} اندیس شروع بازه و {$j$} اندیس پایان بازه است.

\زیرقسمت*{حلقه‌ها}
برای تکرار یک تا تعدادی دستور از دو نوع حلقه استفاده خواهد شد: حلقه {\lr{for}} و حلقه {\lr{while}}. از ساختار حلقه {\lr{for}} زمانی استفاده می‌شود که تعداد تکرار بدنه حلقه از قبل مشخص باشد و از حلقه {\lr{while}} زمانی استفاده می‌شود که تعداد تکرار بدنه حلقه از قبل معلوم نباشد.

تعریف حلقه {\lr{for}} با کلمه کلیدی {\lr{for}} آغاز می‌شود. سپس مقدار اولیه شمارنده حلقه به متغیر شمارنده حلقه انتساب داده می‌شود و پس از کلمه کلیدی {\lr{to}} مقدار نهایی شمارنده حلقه مشخص می‌شود. بعد از تعریف سرآیند حلقه، بدنه حلقه تعریف می‌دهد و در نهایت عبارت {\lr{end for}} پایان حلقه را نشان می‌شود. در ادامه شکل کلی تعریف حلقه {\lr{for}} نشان داده شده است.
\begin{latin}
\begin{algorithmic}[1]
\For{$\mathit{counter}=\mathit{startValue} \To \mathit{endValue}$}
	\State	\Comment{body of for loop}
\EndFor
\end{algorithmic}
\end{latin}
با هر بار اجرای این حلقه یک واحد به متغیر شمارنده حلقه افزوده می‌شود و بدنه حلقه تا زمانی اجرا می‌شود که شرط {$\mathit{counter} \leqslant \mathit{endValue}$} برقرار باشد. اگر بخواهیم شمارنده حلقه به جای افزایش، کاهش یابد آنگاه به جای کلمه کلیدی {\lr{to}} از کلمه کلیدی {\lr{downto}} استفاده می‌شود.

تعریف حلقه {\lr{while}} با کلمه کلیدی {\lr{while}} آغاز می‌شود. سپس یک عبارت منطقی قرار می‌گیرد و تا زمانی که عبارت منطقی برقرار باشد بدنه حلقه اجرا می‌شود. خاتمه حلقه {\lr{while}} نیز با عبارت {\lr{end while}} نشان داده می‌شود.
\begin{latin}
\begin{algorithmic}[1]
\While{$booleanExpression$}
	\State	\Comment{body of while loop}
\EndWhile
\end{algorithmic}
\end{latin}

\زیرقسمت*{دستورات شرطی}
برای شروع یک دستور شرطی از کلمه کلیدی {\lr{if}} استفاده می‌شود و در ادامه یک عبارت منطقی آورده می‌شود. اگر عبارت منطقی درست باشد دستورات بخش اول و در غیر این صورت دستورات بخش دوم اجرا می‌شوند. خاتمه تعریف دستور شرطی نیز با  عبارت {\lr{end if}} نشان داده می‌شود. برای تعریف یک دستور شرطی از قالب کلی زیر استفاده می‌شود.
\begin{latin}
\begin{algorithmic}[1]
\If{$booleanExpression$}
	\State	\Comment{statements to be executed when $booleanExpression$ is \const{true}}
\Else
	\State	\Comment{statements to be executed when $booleanExpression$ is \const{false}}
\EndIf
\end{algorithmic}
\end{latin}
وجود بخش {\lr{else}} اجباری نیست و این یعنی اگر این بخش وجود نداشته باشد و شرط دستور شرطی برقرار نباشد آنگاه بدنه دستور شرطی اجرا نخواهد شد.

\زیرقسمت*{دستور {\lr{return}}}
با اجرای دستور {\lr{return}} اجرای زیربرنامه بلافاصله پایان می‌یابد. از این دستور در صورت نیاز برای بازگرداندن مقدار یا مقادیری به عنوان خروجی یک تابع نیز می‌توان استفاده کرد. در ادامه شکل‌های مختلف دستور {\lr{return}} آورده شده است.
\begin{latin}
\begin{algorithmic}[1]
\State	\Return
\State	\Return result
\State	\Return (result1, result2, $\ldots$ , resultN)
\end{algorithmic}
\end{latin}
در صورتی که شکل خروجی تابعی مانند {\bcall{Func}{$A,B$}} مانند حالت سوم دستور {\lr{return}} باشد آنگاه از شکل زیر برای دریافت تمامی خروجی‌های آن استفاده خواهد شد. با اجرای دستور زیر مقدار {\lr{result1}} در {\lr{var1}} قرار می‌گیرد، {\lr{result2}} در {\lr{var2}} قرار می‌گیرد و به همین ترتیب تا {\lr{resultN}} که در {\lr{varN}} قرار می‌گیرد.
\begin{latin}
\begin{algorithmic}[1]
\State	var1, var2, $\ldots$ , varN$\gets$\bcall{Func}{$A,B$}
\end{algorithmic}
\end{latin}
\زیرقسمت*{عملگرها}
عملگرهای منطقی مورد استفاده در الگوریتم‌ها عبارت‌اند از {$<$}، {$>$}، {$\leq$}، {$\geq$}، {$\neq$} و {$\isequal$} که از عملگر آخر برای بررسی تساوی دو مقدار استفاده می‌شود.

برای ترکیب عبارات منطقی از عملگرهای {\lr{and}}، {\lr{or}} و {\lr{not}} استفاده می‌شود. جهت انتساب مقداری به یک متغیر از عملگر {$=$} استفاده می‌شود.

برای انجام عمل جمع از نماد {$+$}، عمل تفریق از نماد {$-$}، عمل ضرب از نماد {$\times$}، عمل تقسیم صحیح از نماد {$\div$} و عمل تقسیم اعشاری از نماد {$\rdiv$} استفاده می‌شود. همچنین از عملگر {$\bmod$} به عنوان عملگر باقیمانده استفاده خواهد شد.

\زیرقسمت*{اشاره‌گرها}
برای تخصیص فضا به یک اشاره‌گر از زیربرنامه {\bcall{New}{}} استفاده می‌شود. برای مثال اگر {$p$} یک اشاره‌گر باشد و بخواهیم به آن فضا اختصاص دهیم باید زیربرنامه {\bcall{New}{}} را به صورت {\bcall{New}{$p$}} فراخوانی کنیم. همچنین برای آزادسازی فضای اختصاص یافته به یک اشاره‌گر از زیربرنامه {\bcall{Free}{}} استفاده می‌شود.

اگر فرض کنیم {$p$} یک اشاره‌گر باشد که به یک ساختار\پانوشت{منظور از ساختار، چیزی مانند {\lr{struct}} در زبان {\lr{C}} یا {\lr{record}} در زبان پاسکال است.} اشاره دارد و این ساختار دارای فیلدی به نام {\id{next}} است آنگاه از قالب {$\attrib{p}{next}$} برای دسترسی به محتوای فیلد {\id{next}} استفاده می‌شود.

برای بیان مقدار تهی در زمان کار با اشاره‌گرها از ثابت {\const{null}} استفاده می‌شود.