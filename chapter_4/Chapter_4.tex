\chapter{صف و پشته}

\سوال جمع دو عدد صحیح بسیار بزرگ مانند {$18274364583$} و {$8129489165026$} را در نظر بگیرید. چنین اعدادی را نمی‌توان در متغیرهایی از نوع {\lr{int}} در زبان {\lr{C}} و یا {\lr{integer}} در زبان پاسکال ذخیره کرد زیرا این متغیرها توانایی ذخیره‌سازی چنین اعداد بزرگی را ندارند. برای جمع این اعداد می‌توان به آنها به عنوان دنباله‌ای از ارقام نگاه کرد. سه پشته در نظر می‌گیریم. ارقام عدد اول را در پشته اول و ارقام عدد دوم را در پشته دوم قرار می‌دهیم. از پشته سوم به عنوان یک پشته کمکی برای محاسبه حاصل جمع استفاده می‌کنیم. با در نظر گرفتن ایده‌ی مطرح شده الگوریتمی ارائه دهید که دو عدد بسیار بزرگ را دریافت کرده و جمع آن دو را به عنوان خروجی محاسبه کند.

\سوال شعر زیر که متعلق به لوییز کارول است را در نظر بگیرید:

\begin{latin}
Round the wondrous globe, I wander wild,\\
Up and down-hill---Age succeeds to youth---\\
Toiling all in vain to find a child\\
Half so loving, half so dear as Ruth.
\end{latin}

این شعر به فردی به نام روث تقدیم شده است. تقدیم شدن این شعر به فرد نامبرده را می‌توان علاوه بر کلمه آخر خط آخر شعر، از کنار هم قرار دادن حرف اول هر خط از شعر در کنار هم نیز فهمید. با استفاده از ساختمان داده صف الگوریتمی بنویسید که این نوع از اشعار را خط به خط از ورودی خوانده و نام فردی که شعر به آن تقدیم شده است را نمایش دهد.

\سوال یک پشته شامل تعدادی عدد در اختیار است. در قالب تعدادی مثال توضیح دهید که چگونه می‌توان با استفاده از هر یک از موارد زیر ترتیب عناصر این پشته را وارونه کرد.
\شروع{فقرات}
\فقره با استفاده از دو پشته اضافی
\فقره با استفاده از یک صف
\فقره با استفاده از یک پشته اضافه و یک متغیر
\پایان{فقرات}

\سوال یکی از امکانات ویرایشگرهای متن حرفه‌ای، امکان تشخیص عدم توازن تعداد پرانتزهای باز و بسته در یک رشته و یا تودرتویی نادرست پرانتزها است. برای مثال رشته {\lr{``((())())()''}} شامل تعداد متوازنی از پرانتزهای باز و بسته است که به درستی تودرتو شده‌اند اما رشته‌‌هایی مانند {\lr{``)()(''}} و {\lr{``(()''}} رشته‌های درستی نیستند. با استفاده از پشته الگوریتمی بنویسید که رشته‌ای شامل تعدادی پرانتز را دریافت کرده و اگر تعداد پرانتزها متوازن بود و به درستی تودرتو شده بودند مقدار {\const{True}} را به عنوان خروجی بازگرداند.

\سوال به رشته‌ای مانند {\lr{‍‍``NADEDAN''}} یک رشته پالیندروم\پانوشت{\lr{palindrome}} گفته می‌شود زیرا این رشته چه از چپ به راست و چه از راست به چپ خوانده شود نتیجه یکسان خواهد بود. با استفاده از تعداد ثابتی پشته، صف و متغیر کمکی الگوریتمی بنویسید که رشته‌ای را به صورت کاراکتر به کاراکتر از ورودی خوانده و مشخص کند آیا رشته ورودی پالیندروم است یا خیر.

\سوال کارآمدترین روش برای پیاده‌سازی دو پشته در یک آرایه به چه روشی قابل انجام است؟ روش پیشنهادی باید به گونه‌ای باشد که تا زمانی که تمام خانه‌های آرایه پر نشده است بتوان عنصری را در یکی از دو پشته قرار داد و با پیغام پر بودن پشته مواجه نشد. شبه کد زیربرنامه‌های {\bcall{Push}{}} و {\bcall{Pop}{}} را برای چنین ساختار داده‌ای بنویسید.

\سوال یک پشته خالی در اختیار است. اعداد ۱ تا ۶ نیز در ورودی قرار دارند. اعمال زیر بر روی پشته قابل انجام هستند:
\شروع{فقرات}
\فقره {\bcall{Push}{}}: کوچکترین عدد موجود درورودی را برداشته و وارد پشته می‌کند.
\فقره  {\bcall{Pop}{}}: عنصر بالای پشته را خارج کرده و در خروجی چاپ می‌کند.
\پایان{فقرات} 

کدامیک از خروجی‌های زیر را نمی‌توان با هیچ ترتیبی از اجرای اعمال {\bcall{Push}{}} و {\bcall{Pop}{}} به دست آورد. (خروجی‌های زیر را از چپ به راست بخوانید.)
\شروع{شمارش}
\فقره {$123564$}
\فقره  {$324651$}
\فقره  {$215346$}
\پایان{شمارش} 

\سوال الگوریتمی غیر بازگشتی بنویسید که تمام {$n!$} جایگشت‌ اعداد {$\lbrace 1,2,\ldots ,n\rbrace$} را تولید کند.